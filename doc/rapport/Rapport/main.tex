% !TeX root = main.tex
% !TeX spellcheck = fr_FR
\documentclass[12pt,a4paper,twoside]{article}
\usepackage[scaled]{helvet}
% Packages and macros
\usepackage[T1]{fontenc}
%\usepackage[utf8]{inputenc}
\usepackage{mathptmx}
\usepackage{pgf}
\usepackage{pgfpages}
\usepackage{parallel}
\usepackage{siunitx}
\usepackage{booktabs}
\usepackage{fancyhdr}
\usepackage{datetime}
\usepackage{multicol}
\usepackage{enumerate}
\usepackage{pifont}
\usepackage{amssymb}
\usepackage[export]{adjustbox}
\usepackage[margin=1in]{geometry}
\usepackage[french]{babel}
\usepackage{caption}
\usepackage{tikz}
\usepackage{tabularx}
\usepackage{gensymb}
\usepackage{hyperref}
\usepackage{graphicx}
\usepackage{pdfpages}
\usepackage{caption}
\usepackage{subcaption}
\usepackage{listings}


\newcommand{\source}[1]{\vspace{-11pt} \caption*{\small \textit{Source: {#1}} }}

\usepackage{verbatim}

\newdateformat{monthyeardate}{%
  \monthname[\THEMONTH], \THEYEAR}

\begin{document}
\pagestyle{fancy}
\lhead{PROJET 2221}
\chead {\today}
\rhead{LocalisationSousMarine}

% ------------------------- TITLE PAGE INSERTION ------------------------
\begin{titlepage}
   \begin{center}
        \vspace*{1cm}
        \LARGE
        {\Huge \textbf{Localisation sous-marine}}
        
        \vspace{0.3cm}
        Système de logging pour déplacement de module sous-marin.
            
        \vspace{1.5cm}

        \textbf{Ali Zoubir} \vspace{5mm}
        
        \includegraphics[width=0.3\textwidth]{../LOGO-PROJ.png}

        \vfill
            
        Rapport de projet
            
        \vspace{0.8cm}
     
        \includegraphics[width=0.31\textwidth]{../ETML-ES-LOGO.png}

        Génie électrique\\
        École supérieure\\
        Suisse\\
        \monthyeardate\today
            
   \end{center}
\end{titlepage} 
\clearpage

% --------------------- TABLE OF CONTENTS  ------------------------------- 
\tableofcontents
\clearpage

% ------------------------- CAHIER DES CHARGES --------------------------- 
\pagestyle{fancy}
\lhead{EIND}
\chead {\today}
\rhead{TP0}
\fontsize{16}{16}\selectfont

% ---- TITRE ----
\author{Ali Zoubir }
\date{November 2022}

\pagestyle{fancy}

\lhead{ETML-ES}
\chead {\monthyeardate\today}
\rhead{Localisation sous-marine V0.0}


\onecolumn


\begin{figure}
\begin{minipage}{0.47\textwidth}
\centering
\includegraphics[width=.4\textwidth,left,]{../ETML-ES-LOGO.png}
\end{minipage}

\hfill
\begin{minipage}{0.7\textwidth}
\raggedleft
\LARGE \textbf{Localisation sous-marine\\ 2022, V0.0}
\end{minipage}
\end{figure}

\begin{figure}

\hfill


\begin{minipage}{1\textwidth}
% ---- DESCRIPTION ----
\section{Caractéristiques du projet}

\subsection{Description}
L’objectif de ce projet, et de stocker des données de mesures du déplacement d’un module sous-marin par une centrale inertielle, dans le but de mathématiquement le localiser depuis son point de départ (référence). Ceci, car la localisation sous-marine n’est pas une tâche aisée due aux différentes contraintes de communication sous-marine notamment que les ondes électromagnétiques ne se propagent pas facilement.
\end{minipage}

\begin{minipage}{1\textwidth}


\subsection{Aperçu}
    \begin{itemize}
        \item	Sauvegarde d’un set de donnée chaque 100ms.
        \item	Profondeur d’utilisation maximum, de 60m.
        \item	2 heure de logging dans carte SD.
        \item	Sensing sur 9 axes :
        \subitem Mesures [Il est souhaitable que les capteurs choisis aient une faible dérive] ;
        \subsubitem Accéléromètre 3-axes. 
        \subsubitem	Gyroscope 3-axes.
        \subsubitem	Magnétomètre 3-axes. 
        \subsubitem	Senseur de température
        \subsubitem	Profondimètre [0->10bar] [Res 1/10]
        \subsubitem	3 à 5 slots libres MikroE pour autres mesures. 
        \item Possibilité de sauvegarder la localisation de points d’intérêts par :
        \subitem Bouton de sauvegarde [A définir : Magnétique, Optique, Mécanique ou autre].
        \item Batterie, autonomie minimum de 2 heures [~10°].
        \item Charge de la batterie par connecteur USB.
        \item (Optionnel) Lecture des données par connecteur USB (Interfaçage électronique, software optionnel dans cette version).
        \item (Optionnel) Interface LED ou petit écran.\\
    \end{itemize}


\end{minipage}

\end{figure}


\clearpage





% ---- TACHES ----
\subsection{Tâches à réaliser}
Développement et intégration d’un PCB avec capteurs et logging sur carte SD dans une lampe de plongée étanche.
    \begin{itemize}
        \item[•] Développement schématique 
        \subitem- Fonctionnement MCU.
        \subitem-	Périphériques de mesures et de sauvegarde / Bus de communication.
        \subitem-	Gestion batterie 
        \item[•]	Routage pour intégration dans boitier de lampe de plongée 200x45mm.
        \item[•]	Programmation mesure et sauvegarde chaque 100ms.
        \subitem-	Configuration MCU.
        \subitem-	Configuration des périphériques de mesure pour 9-DOF.
        \subitem-	Configuration des périphériques de sauvegarde (Carte SD).
        \subitem-	Configuration et communication avec l'interface.
        \subitem-	Communication et traitement des données mesurées.
\end{itemize}


% ---- SChema de principe ----
\begin{figure}[hb]
    \includegraphics[width=.65\textwidth, center,]{../CDC/Figures/pre-etude.drawio.png}
    \caption{Schéma de principe}
    \source{Auteur}
\end{figure}
\clearpage

% ---- Description des blocs ----
\subsection{Description des blocs}
\begin{enumerate}
    \item \textbf{Carte SD :}\\
    Stockage des données de mesures chaque 100ms, cœur du projet.
    \item \textbf{Accéléromètre-gyroscope-magnétomètre :}\\
    Lecture des données individuelles brute ainsi que de fusion des capteurs, pour mesurer les déplacements sur 9 degrés de libertés.
    \item \textbf{Profondimètre :}\\
    Mesure la pression pour déduire la profondeur, afin de corroborer les autres mesures des capteurs.
    \item \textbf{Real time clock :}\\
    Permet de sauvegarder la temporalité du set de mesure dans la carte SD.
    \item \textbf{Affichage :}\\
    Affichage LED ou écran, pour affichage pas encore définis (ex. Profondeur, état batterie…)
    \item \textbf{Bouton sauvegarde :}\\
    Permet la mise en valeur d’un set de mesure. La forme de ce bouton n’est pas encore définie. Il sera peut-être fusionné avec le bouton ON/OFF.
    \item \textbf{Bouton ON/OFF :}\\
    Permet d’allumer ou d’éteindre le système.
    \item \textbf{Batterie :}\\
    Batterie du système, technologie à définir dans la pré-étude. 
    \item \textbf{COM. USB :}\\
    Permet de charger les batteries. Il faudra également prévoir dans cette version l’interface électronique pour la lecture de la carte SD directement par le port USB.
    \item \textbf{Microcontrôleur :}\\
    Lis et traite les valeurs des capteurs, sauvegarde dans la carte SD...
\end{enumerate}


\clearpage

% ---- JALONS ----
\subsection{Jalons principaux}
\begin{figure}[h!]
    \centering
    \includegraphics[width=.9\textwidth,center,]{../CDC/Figures/Tab-Jalons-PROJ.PNG}
    \caption{Jalons principaux}
    \label{fig:Jalons}
\end{figure}

\subsection{Livrable}
\begin{itemize}
    \item[•] Les fichiers sources de CAO électronique des PCB réalisés
    \item[•] Tout le nécessaire à fabriquer un exemplaire hardware de chaque :
    \item[•] fichiers de fabrication (GERBER) / liste de pièces avec références pour commande / implantation
    \item[•] Prototype fonctionnel
    \item[•] Modifications / dessins mécaniques, etc
    \item[•] Les fichiers sources de programmation microcontrôleur (.c  / .h)
    \item[•] Tout le nécessaire pour programmer les microcontrôleurs (logiciel ou fichier .hex)
    \item[•] Un calcul / estimation des coûts
    \item[•] Un rapport contenant les calculs - dimensionnement de composants - structogramme, etc.
\end{itemize}

\clearpage


% ------------------- PRE-ETUDE ---------------------- ----
% ------------------------- PRELIMINARY TASK -----------------------------
\section{Pré-étude}
L'objectif de cette pré-étude, est de se pencher sur le fonctionnement plus fondamental du système, faire des petits dimensionnements ainsi que de survoler différents aspects techniques liés au projet.
% ---------------- Subsection 1 ----------------
\subsection{Fonctionnement du système} \label{ssec:num01}
{
\subsubsection{Schéma bloc}

\begin{figure}[h]
    \centering
    \includegraphics[width=.9\textwidth]{Figures/Schema-bloc-LocalisationSousMarin.drawio}
    \caption{Schéma bloc du module}
    \source{Auteur}
    \label{fig:SchemaBloc}
\end{figure}

\clearpage

\underline{Capteurs :} 
\\ Les différents capteurs sont interfacés sur le même bus, et ont comme master le microcontrôleur en communication bidirectionnel, afin d'à la fois configurer les registres des périphériques et de lire leurs mesures.\\

\underline{Carte SD :}
\\ La carte SD est interfacée en SPI et va contenir les données des différents capteurs ainsi que leurs éventuels flags d'importance (sauvegarde), sa taille sera dimensionnée ultérieurement.\\

\underline{Port USB \& charge :}
\\ Un port USB est présent, afin charger les batteries par un IC de gestion de charge connecté directement au 5V. De plus le port USB est communiquant avec le microcontrôleur par un driver FTDI, afin d'éventuellement ajouter un système de lecture de la carte SD, directement par USB. Ceci dans cette version ou une ultérieure. Le port USB pourrait aussi servir a fixer la référence de la RTC. \\

\underline{Bouton multifonction :}
\\ Sachant qu'un bouton étanche est déjà présent sur le module, l'exploiter en tant que bouton multifonction est une solution ergonomique pour ne pas mettre en péril l'étanchéité globale. Ce bouton ferait office de ON/OFF et de "sauvegarde" de point d'intérêt. Pour se faire, le bouton contrôlerait par un transistor de commutation l'alimentation du système, puis lors de l'allumage du microcontrôleur, le MCU prendrait la relève en maintenant le système allumé a sont tour, permettant ainsi de lire le bouton et de sur une pression longue déconnecter l'alimentation.\\

\underline{Affichage :}
\\ L'affichage permettra de visualiser différentes données, dont les plus importantes tel que la pression ou le statut de la batterie. \\ La forme de l'affichage est encore a définir selon la mécanique du module, mais le plus élégant, serait l'utilisation d'un petit écran OLED.\\

\underline{Capteur de pression :}
\\ Le capteur de pression devra avoir un contact direct avec l'eau, cela impliquera de la mécanique et de la gestion d'étanchéité. Une autre possibilité aurait été de mesurer optiquement la déformation du boîtier pour en déduire la pression, mais la complexité est trop importante.\\
}

\clearpage

% ----------------Subsection 2 ----------------
\subsection{Choix des composants importants} \label{ssec:num02}
{

\subsubsection{Senseur absolu}
{
    Pour le senseur absolu, il existe des IC permettant directement de faire la fusion des senseurs (\textbf{Accéléromètre, gyroscope, magnétomètre et thermomètre}), ce qui épargne toute une phase de calcul chronophage, en permettant directement de lire les \textbf{quaternion, angles de Euler, vecteurs de rotations, cap de direction etc...} directement sur le composant. Il existe différents IC dont deux ce sont montrés très intéressants, le \textbf{BNO85} et le \textbf{BNO55}, les deux étant PIN-Compatibles, j'ai décidé d'opter pour le \textbf{\underline{BNO055}}\footnote{K:/ES/PROJETS/SLO/2221\_LocalisationSousMarine/doc/composants/9DOF-BNO055}.
    
    \begin{figure}[h]
    \centering
    \includegraphics[width=.4\textwidth]{Figures/BNO055-Illustration}
    \caption{Schéma bloc du module}
    \source{https://www.mouser.ch/new/bosch/bosch-bno55-sensor/}
    \label{fig:SchemaBloc}
    \end{figure}
    
    Sachant que la brazure de ce type de boîtier est compliquée et également dans un but de simplification du projet, j'ai décidé d'utiliser les cartes d'évaluation d'adafruit \textbf{N°: 4646} qui ont des connections bergs ainsi que tous les composants externes passifs déjà montés. \\
    
    \underline{Caractéristiques importantes :} \\
    
    \begin{tabular}{l l l l}
        Résolution gyroscope & : & 16 & [bits] \\
        Résolution accéléromètre & : & 14 & [bits] \\
        Résolution magnétomètre & : & $\sim$0.3 & [$\mu$T] \\
        $I_{DD}$ & : & 12.3 & [mA] \\
        Dérive de température & : & $\pm$ 0.03 & [\%/K] \\ 
        Dérive accéléromètre & : & 0.2 & [\%/V] \\
        Dérive gyroscope & : & <0.4 & [\%/V]
    \end{tabular} \\
    Nous allons par la suite voir sur la figure \ref{fig:BnoOut}, quelles données du BNO055 sont disponibles ainsi que leurs tailles mémoires.
    
    \begin{figure}[h] 
        \centering
        \includegraphics[width=.68\textwidth]{Figures/DATAS-BNO055}
        \caption{Donnée de sortie de l'IC (43 bytes)}
        \source{ https://cdn-shop.adafruit.com/datasheets/BST\_BNO055\_DS000\_12.pdf }
        \label{fig:BnoOut}
    \end{figure}
}

\clearpage

\subsubsection{Capteur de pression}
{
Pour le capteur de pression, une modification mécanique du boîter sera très probablement nécessaire. J'ai pu trouver un capteur correspondant aux caractéristiques demandée du projet, celui-ci est plutôt générique et peut communiquer en I2C : \\
\textbf{PTE7300-14DN-0B016BN}

\begin{figure}[h] 
    \centering
    \includegraphics[width=.4\textwidth]{Figures/Capteur-pression}
    \caption{Illustration capteur de pression}
    \source{Distrelec, PTE7300-14DN-0B016BN}
    \label{fig:CaptPress}
\end{figure}

L'avantage avec le capteur ci-dessus est le système hermétique pour le trou, un autre capteur peut être utilisé lors de l'étude, néanmoins la modification mécanique étant probablement inévitable, le système de vissage de la figure \ref{fig:CaptPress} est intéréssant.

}

\subsubsection{Affichage}
{
    Pour l'affichage, je vais essayer d'opter pour un petit afficheur OLED, en gardant la possibilité en cas de de complication lors de l'étude, l'utilisation de simples LEDS d'indications.
    \\
    Il existe plusieurs affichages OLED rond petits formats, sur lesquels je me pencherais plus en détail lors de l'étude.
    


}

\newpage
\subsubsection{Carte SD} \label{sssec:CarteSD}
{
    \underline{Taille mémoire :} \\
    Afin de dimensionner la taille de stockage de la carte SD, il faut utiliser les différentes caractéristiques du projet. Normalement la taille de la carte SD n'est clairement pas un problème, sachant que seulement du texte est enregistré et que les tailles mémoires disponibles peuvent être très élevées. Néanmoins il est intéressant de faire le dimensionnement pour connaître le minimum, et pour éventuellement adapter le projet avec d'autres systèmes de mémorisation.  \\
    Où : \vspace{+14pt} \\
    \begin{tabular}{l l ll|l}
       $ T_{rec} $ & = &  $7200'000$ & $[ms]$ & Temps a enregistrer \\
       $ T_{ech}$ & = & $100$  & $[ms]$ & Temps d'un échantillon \\
       $ S_{mes} $ & = & $43$ & $[bytes]$ & Taille de toutes les données de mesures  \\
       $ S_{timestamp} $ & = & $\sim$23 & $[bytes]$ & Taille de l'information de temporalité  \\
       $ S_{flag} $ & = & $ 1 $ & $[bytes]$ & Taille de l'indication d'importance 
    \end{tabular}
    \vspace{+14pt}
    \\
    Nombre de mesure a effectuer : \\ 
    \begin{equation} \label{equ:NbMes}
        Nb_{mesures} = \frac{T_{rec}}{T_{ech}}
    \end{equation} 
    D'après (\ref{equ:NbMes}), nous avons un nombre de mesure de 72'000.  \vspace{+8pt} \\
    
    Taille minimum : \\
    \begin{equation} \label{equ:TailleMin}
        Taille_{min} = Nb_{mesures} * (S_{mes}+S_{timestamp}+S_{flag}) 
    \end{equation}
    D'après (\ref{equ:TailleMin}), la taille mémoire minimum doit être de \textbf{$\sim$5MB}. \vspace{+8pt} \\ 
    Nous pouvons donc constater que pour une utilisation standard de 2h, la mémoire occupée est très faible, d'où l'intérêt de sauvegarder dans la carte SD la date, afin de pouvoir faire plusieurs "expéditions" en "une fois", sans avoir à vider la carte.
}

\newpage
\subsubsection{Real Time Clock}
{
    L'objectif de la RTC, est de donner l'information de la temporalité de la mesure (timestamp), afin de lors du traitement des donnée avoir accès à ce paramètre. \\
    Sachant que l'échantillonnage des mesures est de 100ms, la RTC devrait permettre cette résolution. Néanmoins une autre information importante, comme mentionnée lors de la section \ref{sssec:CarteSD}, est la date de la mesure, afin de permettre plusieurs expéditions par utilisation de la carte. \vspace{+8pt} \\
    J'ai donc décidé d'utiliser une RTC pour l'heure grossière de départ (Année, date, heure, minute, seconde) et les compteur du MCU pour faire le delta entre chacune des mesures en ms. \\

    La RTC devra pouvoir tenir le minimum de 2 heure d'utilisation, à cette fin, la batterie LI-ION déjà présente sera suffisante. \\ 
    La RTC devra avoir une faible consommation, le calendrier ainsi qu'une bonne précision. A cette fin, la RTC \textbf{S-35390A-T8T1G} est assez générique et possède une bonne documentation.
    
    \begin{figure}[h] 
        \centering
        \includegraphics[width=.4\textwidth]{Figures/RTC}
        \caption{Illustration de la RTC}
        \source{ https://www.digikey.com/en/products/detail/ablic-inc/S-35390A-T8T1G/1628383 }
        \label{fig:RTC}
    \end{figure}    
    
}
\clearpage

\subsubsection{Microcontrôleur}
{
    Le microcontrôleur devra avoir un nombre suffisant de communications, sachant que beaucoup sont présentes dans le projet (\textbf{I2C, SPI, UART...}), ce qui signifie un nombre de pattes élevées. 
    
    
    Des calculs peuvent aussi être nécessaire, si il s'avère qu'il faille faire une traitement des données préliminaire, il faudrait donc opter pour un MCU 32bits si possible.


    La famille PIC est celle standardisée par l'école supérieure, c'est donc pour cette famille-ci que je vais opter.
    \vspace{+12pt} \\

    \begin{figure}[h] 
        \centering
        \includegraphics[width=.4\textwidth]{Figures/PIC32MX795F512L-V7X-Regular}
        \caption{Illustration du modèle MCU du kit ETML-ES}
        \source{ https://www.microchip.com/en-us/product/PIC32MX795F512L }
        \label{fig:MCU}
    \end{figure}
    
}

\clearpage

\subsubsection{Batterie, charge et régulation}
{

Pour la technologie de batterie, en utilisation sous-marine, j'ai trouvé ce tableau de comparaison :

\begin{figure}[h]
    \centering
    \includegraphics[width=.56\textwidth]{Figures/PowerSystemsComparison}
    \caption{Comparaison des technologies de batteries}
    \source{Power Systems for Autonomous Underwater Vehicles\cite{bradley_power_2001}}
    \label{fig:BatteriesComparaisons}
\end{figure}

Pour des raisons de praticité et étant-donné la documentation plus importante, j'ai décidé d'utiliser la technologie \textbf{LI-ION} : \\

\centering
\begin{tabular}{c|c}
    Avantages &  Inconvénient\\
    \hline
    Haute densité d'énergie & Risque d'éclatement \\
    Poids léger & Risque d'enflammement avec l'eau \\ 
    Haute durée de vie & Sensible a la température \\
    Charge rapide & Décharge complète altérante \\
\end{tabular}
\vspace{+8pt}
\\ Malgré les risque dûs au contact de l'eau (\textbf{Enflammement, éclatement...}) la technologie LI-ION est souvent utilisée pour les application sous-marines dû a ses différents avantages, c'est pour cela que j'opterais pour cette technologie. 

}

}

\newpage
\subsection{Estimation des coûts} \label{ssec:EstPrix}
{
    Ici je vais me baser sur les composants que j'ai pu trouver et estimer le coût moyen de ceux-ci, c'est a titre purement indicatif, (les prix sont généralement estimés a la hausse).
    \vspace{+12pt}
    
    \begin{center}
        \begin{tabular}{l|l}
            Composant & Estimation \\
            \hline
            Profondimètre & 70.- \\
            Centrale inertielle & 35.- \\
            RTC & 5.- \\
            Microcontrôleur & 5.- \\
            Carte SD & 20.- \\
            Affichage OLED & 45.- \\
            FTDI & 4.- \\
            Batterie LI-ION & 20.- \\
            IC chargeur & 4.- \\
            Traco-power 3.3V & 10.- \\
            PCB & 40.- \\
            \hline
            \hline
            Total & 258.-
        \end{tabular} 
    \end{center}
	

    L'estimation des prix étant plutôt élevée, des économies peuvent être très facilement réalisées, en changeant l'affichage OLED pour des LEDS ou en modifiant le PCB (Le simplifier ou changer de fournisseur (eurocircuit)).

}

\newpage


\subsection{Conclusion et perspectives} \label{ssec:PreeConc}
{

J'ai pu lors de cette pré-étude, établir le fonctionnement global du système, choisir certaines technologies et composants importants, ainsi que pu procéder a certains dimensionnements utiles quant au futur développement.


Par la suite, je vais affiner les différents éléments abordés lors de la pré-étude, effectuer le développement plus détaillé de chacun des blocs et réaliser la schématique du projet.


Lors de la pré-étude, je n'ai pas eu accès au boîtier mécanique du projet, ce qui a restreint mon champs d'action lors de certains dimensionnement, tandis que pendant l'étude j'aurais accès a celui-ci, ce qui risque d'impacter/modifier certains aspect fixés lors des section antérieures.


Je suis très intéressé par le projet et me réjouis grandement de poursuivre son développement.

}

\clearpage


% ---------- DÉVELOPPEMENT SCHÉMATIQUE --------------------
% ------------------------- MAIN TASK ---------------------------------
\section{Développement schématique}

\subsection{Choix des composants} \label{ssec:num32}
{
	\subsubsection{Microcontrôleur}
	Lors de la recherche de composants, j'ai décidé d'utiliser l'un des PIC32 standards de l'ES :
	\textbf{PIC32MX130F064D-I/PT}.
		
	\begin{figure}[h]
		\centering
		\includegraphics[width=1\linewidth]{Figures/Dev-SCH/PIC32-choisi}
		\caption{Périphériques disponibles du PIC}
		\label{fig:pic32-choisi}
		\source{PIC32MM0256GPM064 family datasheet}
	\end{figure}
	
	Nous pouvons constater sur la figure \ref{fig:pic32-choisi} que les critères minimaux de mon projet sont respectés :
	
	\begin{center}
		\fbox{\textit{1 - I2C}} \fbox{\textit{1 - SPI}} \fbox{\textit{1 - UART}} \fbox{\textit{1 - RTCC}}
	\end{center}
}

\clearpage
\subsection{Dimensionnements} \label{ssec:num31}
{
	\subsubsection{Vue d'ensemble schématique} \label{sssec:SchemaBloc}
	{
		\begin{figure}[th]
			\centering
			\includegraphics[width=1.1\linewidth]{Figures/Dev-SCH/schemaBloc}
			\caption{Schéma bloc de la schématique}
			\source{Auteur}
			\label{fig:schemablocSCH}
		\end{figure}
		
		Nous pouvons constater sur la figure \ref{fig:schemablocSCH} la structure des différents blocs du schéma : \vspace{+5mm}
		
		\begin{tabularx}{18cm}{|X|X|}
			\hline
			Bloc & Description \\
			\hline
			\hline
			Power & Contient les différents régulateurs du système, ainsi que la gestion de charge de la batterie. \\
			\hline
			MCU & Contient l'intelligence du système, avec le microcontrôleur ainsi que tous ses composants passifs associés. \\
			\hline
			Peripherals & Périphériques du système : Carte-SD, Centrale inertielle, Capteur de pression, Slots MikroE. \\ 
			\hline
			Interface & Connecteur USB avec convertisseur serial (FTDI) et tous les composants passifs de sécurité. \\
			\hline
		\end{tabularx}
	}

	\clearpage
	\subsubsection{Chargeur de batterie} \label{sssec:BatCharger}
	{
		
	}
	
	\subsubsection{Régulateur 3.3V} \label{sssec:Reg3V3}
	{
		
	}
	
	\subsubsection{Régulateur 5V} \label{sssec:Reg5V}
	{
		
	}
	

}

\clearpage

% ---------- DÉVELOPPEMENT PCB ----------------------------- 
% ------------------------- MAIN TASK ---------------------------------
\section{Développement du PCB}

\subsection{Bill of materials} \label{ssec:BOM}
{
	La BOM complète est disponible dans les répertoires du projet, voici un extrait des prix des composants importants :
	
	\begin{figure}[h]
		\centering
		\includegraphics[width=0.7\linewidth]{Figures/Prix_Composants}
		\caption{Prix des composants}
		\label{fig:prixcomposants}
	\end{figure}
	
}
\clearpage

\subsection{Mécanique du projet} \label{ssec:mechProjet}
{
	Afin d'installer le PCB dans le boitier de la lampe, les deux tiges internes du boitier peuvent avec des attaches servir de support. Pour se faire, imprimer une pièce a visser en 3d ou en utilisant simplement des brides, pourrait permettre de maintenir la carte dans le boitier. Il faudra donc mettre des trous dans le PCB pour permettre des visses ou des brides.
	
	
	\subsubsection{Considérations mécaniques} \label{ssec:RestrictionMech}
	\paragraph{Tige conductrice :} Sachant que les tiges de support de la lampe sont conductrices, il faut donc prévoir une zone sans composant, sans cuivre apparent et si possible sans pistes sur les bords de la couche \textit{bottom}.
	\paragraph{Carte SD : } La carte SD requiert un support relativement grand et un espace doit être prévu pour pouvoir insérer/retirer la carte facilement et sans qu'elle dépasse trop du PCB.
	\paragraph{Centra inertielle :} La centrale inertielle, se connecte via des bergs femelles et vas par conséquent prendre de la place en hauteur, ce qui doit être considéré.
	\paragraph{Slot MIKROE :} Un slot mikroe est présent dans le projet et pour être implémenté, vas requérir un allongement mécanique du bouton de la lampe, pour gagner de la place. Cette pièce doit être produite et usinée, car elle requiert d'être étanche. 
	\paragraph{LED RGB :} La LED en bout du PCB peut exploiter le réfracteur déjà présent de la lampe.
	\paragraph{Connecteur USB :} Pour charger l'appareil, un port USB devrait être disponible au bord du PCB.
}
\clearpage

\subsection{Placement des composants} \label{ssec:placementComp}
{
\paragraph{Alimentation :}
\begin{figure}[h]
	\centering
	\includegraphics[width=0.9\linewidth]{Figures/BasPCB}
	\caption{Placement alimentations}
	\label{fig:baspcb}
\end{figure}

J'ai décidé de placer en bas du PCB les parties d'alimentations du projet. Cette zone comporte :

% Please add the following required packages to your document preamble:
% \usepackage{graphicx}
\begin{table}[h]
	\centering
	\resizebox{\textwidth}{!}{%
		\begin{tabular}{|c|l|l|}
			\hline
			Composant &
			\multicolumn{1}{c|}{Detail} &
			\multicolumn{1}{c|}{Justification} \\ \hline
			USB &
			FTDI, piste 5V, pistes différentiels &
			Bord du PCB pour cablage ergonomique \\ \hline
			Capteur de pression &
			\begin{tabular}[c]{@{}l@{}}Connecteur, Jumper choix alim,\\ Choix mode (courant,tension,i2c)\end{tabular} &
			\begin{tabular}[c]{@{}l@{}}Capteur de pression proche du bord\\ branchement plus simple\end{tabular} \\ \hline
			Carte SD &
			Condos de découplages, pistes SPI &
			Plus grand consommateur \\ \hline
			Gestion batterie &
			\begin{tabular}[c]{@{}l@{}}Régulateur de charge, bouton ON/OFF,\\ régulateur 3.3V\end{tabular} &
			Zone dédiée, proche de la batterie \\ \hline
			Boost 5V (bottom) &
			Circuit de boost 3.3-5V &
			\begin{tabular}[c]{@{}l@{}}Isolé, loin des petits signaux (bruit).\\ Proche du capteur de pression (5V)\end{tabular} \\ \hline
		\end{tabular}%
	}
	\caption{Composants de la zone et justification}
	\label{tab:PlacementBas}
\end{table}
\clearpage

\begin{figure}[h]
	\centering
	\includegraphics[width=0.56\linewidth]{Figures/hautPCB}
	\caption{Placement signaux}
	\label{fig:hautpcb}
\end{figure}

Le microcontrôleur est plus ou moins centrée par rapport aux différents périphériques, afin de minimiser la longueur des pistes des petits signaux. Il y a un bouton de reset, un multiplexeur et un oscillateur externe, au plus proche du MCU.
}
\clearpage

\subsection{Mécanique du PCB}
{

	\begin{figure}[h]
		\centering
		\includegraphics[width=1\linewidth]{Figures/2221_PCB_MechDims_page-0001}
		\caption{Plan mécanique du PCB}
		\label{fig:2221pcbmechdimspage-0001}
	\end{figure}
	La conception du PCB comprend des dimensions spécifiques, avec une longueur de 188.15 mm, une largeur de 40.51 mm et une hauteur de 27 mm. Pour faciliter l'installation et la fixation, cinq trous M3 sont répartis le long du PCB. Cependant, la présence du socle de la pile, qui a une hauteur de 15.9 mm, peut poser un problème, car il peut entraver le placement du PCB.
	Afin de résoudre cette contrainte, une décision a été prise de positionner le socle de la pile au centre du PCB. Cette disposition permet de mieux répartir l'espace disponible et d'éviter que le socle de la pile ne perturbe les autres contraintes mécaniques.
	En examinant une représentation en 3D de la carte, on peut constater que la carte SD simulée s'intègre parfaitement à la surface du PCB et ne dépasse pas ses dimensions.
}

\clearpage
\subsection{Routage} \label{ssec:routage}
{
	\begin{figure}[h]
		\centering
		\includegraphics[height=.67\textheight]{Figures/Pistes_PCB}
		\caption{Pistes du PCB}
		\label{fig:pistespcb}
	\end{figure}
	
	Routage sur 4 couches, chacune possède une orientation de piste : Top-Horizontal, Bottom-Vertical, Layer1-Vertical, Layer2-Vertical. La zone la plus denses est celle du microcontrôleur, c'est aussi où il y a le plus de vias. On peut voir que les pistes du bas sur la figure \ref{fig:pistespcb} sont plus épaisses, c'est parce que ce sont les pistes d'alimentations principales. Le 3.3V a une arborescence en arbre, avec le tronc qui traverse le PCB et les branches qui vont alimenter les différents systèmes, en passant bien d'abord par les condensateur de découplages de chacun. 
	\paragraph{Piste différentiel :} L'USB possède une piste différentiel Data+/Data-, on peut la voir en bas de la figure \ref{fig:pistespcb} (Couche brune).
}
\clearpage


% ---------- DÉVELOPPEMENT SOFTWARE ----------------------------- 
\section{Développement firmware}
Dans cette section, nous allons décrire et expliquer le procédé de programmation du code qui a été implémenté dans le microcontrôleur PIC32MX130F256D.
Le processus de programmation du code dans le PIC32MX130F256D implique plusieurs étapes. Tout d'abord, il est nécessaire de disposer d'un environnement de développement intégré (IDE) adapté à ce microcontrôleur, ici, MPLAB X IDE, avec l'environnement Harmony permettant l'utilisation d'un configurateur graphiques pour les différentes libraires du PIC.

\subsection{Configuration des PINs dans Harmony}
{
	\begin{figure}[h]
		\centering
		\includegraphics[width=.9\textwidth]{Figures/Dev-SOFT/MCU-Altium}
		\caption{Pinning réelles dans altium designer}
		\label{fig:mcu-altium}
	\end{figure}
	\begin{figure}[h]
		\centering
		\includegraphics[width=0.5\textwidth]{Figures/Dev-SOFT/MCU-Harmony}
		\caption{Pinning dans Harmony}
		\label{fig:mcu-harmony}
	\end{figure}
	
	On peut voir que la PIN 20 (SCK) est en haute impédance dans harmony, tandis que la PIN14 qui était supposée être \textit{U2TX} est devenue SCK, ceci étant dû à une erreur : SCK n'est que valable sur la PIN14, il a donc fallut ajouter un fil extérieur pour router Pin14 à Pin20, sacrifiant ainsi la communication USB sur l'UART2, cette modification sera décrite ultérieurement.
	
	
}

\subsection{Configuration des périphériques dans Harmony}
{
	\subsubsection{Timers} 
	
	Deux timers seront utilisés, l'un pour mesurer des attentes en ms et l'autre moins rapide, pour les diverses actions du programmes, avec une interruptions chaque 10ms.
	\begin{table}[h]
		\centering
		\begin{tabular}{|l|l|}
			\hline
			Timer & Temps voulu \\
			\hline
			Timer 1 & 1ms \\
			\hline
			Timer 2 & 10ms \\
			\hline
		\end{tabular}
	\end{table}

	La clock du système a été accélérée a 48MHz, cela dans le but d'accéléré le système dans sa globalité, sachant qu'il y a beaucoup de communication qui nécessite des opérations rapides, que ce soit dans la préparation des buffers ou les différents calculs.
	
	\begin{figure}[h]
		\centering
		\includegraphics[width=0.55\linewidth]{Figures/Dev-SOFT/Timer_config}
		\caption{Configuration dans harmony}
		\label{fig:timerconfig}
	\end{figure}

	\begin{figure}[h]
		\centering
		\begin{subfigure}[b]{0.45\textwidth}
			\centering
			\includegraphics[width=\textwidth]{Figures/Dev-SOFT/Timer1ms}
			\caption{Timer 1, Dimensionnement pour 1ms}
			\label{fig:timer1ms}
		\end{subfigure}
		\hfill
		\begin{subfigure}[b]{0.45\textwidth}
			\centering
			\includegraphics[width=\textwidth]{Figures/Dev-SOFT/Timer10ms}
			\caption{Timer 2, Dimensionnement pour 10ms}
			\label{fig:timer10ms}
		\end{subfigure}
		\hfill
		\caption{Application timer développée par l'auteur}
		\label{fig:appTimer}
	\end{figure}

	\clearpage

	\subsubsection{USART} \label{ssec:Usart}
	Dans le but  de vérifier les données de mesures en temps réelles, simplifiant ainsi le debuggage, une communication sérielle UART a été mise en place. Pour se faire, le périphériques UART1 prévu à la base pour le slot mikroe a été utilisé. Un module USB-to-TTL externe permettra de lire les données via Putty sur un ordinateur.
	
	\begin{figure}[h]
		\centering
		\includegraphics[width=0.7\linewidth]{Figures/Dev-SOFT/ConfigUart}
		\caption{Configuration UART}
		\label{fig:configuart}
	\end{figure}
	
	
	Nous pouvons constater sur la figure \ref{fig:configuart} que l'UART est configuré sans interruption à un bauderate de 115200.

	\subsubsection{Carte SD - SPI} 
	{
	L'utilisation du SPI a été optimisée en choisissant une fréquence de 5 MHz afin de minimiser le temps d'exécution sur le microcontrôleur, sachant que les opérations FAT nécessitent de nombreuses trames. Cependant, j'ai rencontré un problème lié à la clock du SPI. Étant donné la vitesse élevée et les modifications que j'ai dû effectuer, la clock interfère avec le FTDI inutilement, et un fil relie SCK et U2TX, créant ainsi une inductance parasite. Pour résoudre ce problème, j'ai dû ajouter un condensateur de \textbf{33pF} entre SCK et GND, pour stabiliser la communication. Voir la configuration harmony de la carte SD sur la figure \ref{fig:configsdspi}.
	\clearpage
	\begin{figure}[h]
		\centering
		\includegraphics[width=0.7\linewidth]{Figures/Dev-SOFT/ConfigSD_SPI}
		\caption{Configuration du SPI}
		\label{fig:configsdspi}
	\end{figure}
	}

	\subsection{Code}
	Je vais dans cette section décrire le code du projet. Voici la hiérarchie des fichier du projet :
	\begin{figure}[h]
		\centering
		\includegraphics[width=0.79\linewidth]{Figures/Dev-SOFT/ClassesCode}
		\caption{Hiérarchie des fichiers du projet}
		\label{fig:classescode}
	\end{figure}
	
	\clearpage	
	
	\subsubsection{Callbacks}
	{
	Chacun des timers appellent dans leur interruption une fonction appartenante au fichier \textit{app.c} qui contient les actions définies pour chaque lapse de temps fixés. On peut visualiser cela sur le diagramme de la figure \ref{fig:callbacks}.
	
	\begin{figure}[h!]
		\centering
		\includegraphics[width=1\textwidth]{Figures/Dev-SOFT/Callbacks}
		\caption{Interactions des interruptions et des callbacks}
		\label{fig:callbacks}
	\end{figure}

	Les callbacks en C offrent flexibilité, extensibilité et réutilisabilité. Ils permettent d'ajuster dynamiquement le comportement du programme, d'étendre les fonctionnalités et de réutiliser le code. Les callbacks favorisent également l'encapsulation et la personnalisation, améliorant ainsi la modularité et la maintenance du code.	
	}
	
	\clearpage
	\subsubsection{Centrale inertielle BNO055}
	Pour ce qui est de la centrale inertielle BNO055, j'ai utilisé la libraire de BOSCH \footnote{\href{https://github.com/BoschSensortec/BNO055_driver}{Librairie du fabricant}}; Que j'ai configuré en 32bits, puis j'ai créer les fonctions bas-niveau (i2c) et ai faire le liens avec la libraire BOSCH. Le tout dans le fichier BNO055\_support.c.
	
	Pour faire le lien entre la librairie haut-niveau et bas-niveau, j'ai utilisé un pointeur de fonction présent dans la structure de donnée du BNO comme sur le listing \ref{lst:lienPointeur}.
	
\begin{lstlisting}[frame=single, label={lst:lienPointeur}, language=C, caption={Code lien pointeur de fonction}, captionpos=b]
s8 I2C_routine(void)
{
	bno055.bus_write = BNO055_I2C_bus_write;
	bno055.bus_read = BNO055_I2C_bus_read;
	bno055.delay_msec = BNO055_delay_msek;
	bno055.dev_addr = BNO055_I2C_ADDR1;
	return BNO055_INIT_VALUE;
}
\end{lstlisting}

	Voici le code une écriture sur le BNO055 par I2C :
\begin{lstlisting}[frame=single, language=C, caption={Code écriture au BNO055}, captionpos=b, breaklines=true]
s8 BNO055_I2C_bus_write(u8 dev_addr, u8 reg_addr, u8 *reg_data, u8 cnt)
{
	s8 BNO055_iERROR = BNO055_INIT_VALUE;
	u8 array[I2C_BUFFER_LEN];
	u8 stringpos = BNO055_INIT_VALUE;
	array[BNO055_INIT_VALUE] = reg_addr;
	
	i2c_start();
	BNO055_iERROR = i2c_write(dev_addr<<1);
	
	for (stringpos = BNO055_INIT_VALUE; stringpos < (cnt+BNO055_I2C_BUS_WRITE_ARRAY_INDEX); stringpos++)
	{
		BNO055_iERROR = i2c_write(array[stringpos]);
		array[stringpos + BNO055_I2C_BUS_WRITE_ARRAY_INDEX] = *(reg_data + stringpos);
	}
	
	i2c_stop();
	if(BNO055_iERROR-1 != 0)
		BNO055_iERROR = -1;
	else
		BNO055_iERROR = 0;
	return (s8)(BNO055_iERROR);
}
\end{lstlisting}
	Pour ce qui est de l'utilisation de la libraire haut-niveau bno055\_support, voici la préparation et la lecture des données : 
	
	\begin{lstlisting}[frame=single, language=C, caption={Code lecture des données par la librairie}, captionpos=b, breaklines=true]
/* BNO055 Read all important info routine */
bno055_local_data.comres = bno055_read_routine(&bno055_local_data);
/* Delta time */
bno055_local_data.d_time = timerData.TmrMeas - timerData.ltime;
/* Pressure measure */
bno055_local_data.pressure = Press_readPressure();
/* Flag measure value */
bno055_local_data.flagImportantMeas = flagMeas;
	\end{lstlisting}

	\subsubsection{Carte SD}
	La  communication de la carte SD fonctionne sous forme d'une machine d'état non-bloquante, permettant ainsi de s'adapter aux situations de la carte sans bloquer le système pour autant.
	
	\begin{figure}[h]
		\centering
		\includegraphics[width=0.6\linewidth]{Figures/Dev-SOFT/mermaid-diagram-2023-06-14-162436}
		\caption{Machine d'état de la carte SD}
		\label{fig:mermaid-diagram-2023-06-14-162436}
	\end{figure}

	\clearpage
	
	\paragraph{Planification d'une écriture}
	Afin de lancer une écriture d'un set de mesure sur la carte SD, il faut utiliser la fonction \textit{sd\_BNO\_scheduleWrite()} qui vas préparer le buffer d'écriture et modifier l'état de la carte SD : 
	\begin{lstlisting}[frame=single, language=C, caption={Lancement d'une écriture sur la carte SD}, captionpos=b, breaklines=true]
/* Write measures to sdCard */
sd_BNO_scheduleWrite(&bno055_local_data);
	\end{lstlisting}
	
	Les données sont écrite dans un fichier .CSV nommé "MESURES". La forme de la trame est la suivante :
	
	\begin{figure}[h]
		\centering
		\includegraphics[width=1.1\textwidth]{Figures/Dev-SOFT/Trame}
		\caption{Format de la trame}
		\label{fig:trame}
	\end{figure}

	\begin{lstlisting}[frame=single, language=C, caption={Ecriture du buffer}, captionpos=b, breaklines=true]
"%d;%d0;%f;%.4f;%.4f;%.4f;%.4f;%.4f;%.4f;%.4f;%.4f;%.4f;%.4f;%.4f;%.4f;%.4f;%.4f;%.4f;%d;%d;%d;%d;"
	\end{lstlisting}
	
	\paragraph{Exemple trame CSV :} 
	0;372;1.025;-0.2500;0.3800;9.7900;-0.1250;0.1250;-0.0625;-44.8750;35.0625;-7.2500;0.0000;-0.0100;-0.2700;0.0000;-2.1875;-1.5000;16379;320;216;-1;

	
	\subsubsection{Application main}
	On peut visualiser la machine d'état de l'application principale sur la figure \ref{fig:stateapp}.
	\begin{figure}[h]
		\centering
		\includegraphics[width=0.28\linewidth]{Figures/Dev-SOFT/StateApp}
		\caption{Machine d'état application}
		\label{fig:stateapp}
	\end{figure}

	\clearpage
	
	\paragraph{Fonctionnement APP\_STATE\_LOGGING :} Une fois que l'application est en mode logging, le fonctionnement est décris sur la figure \ref{fig:appstatelogging} sous forme de flowchart.
	
	\begin{figure}[h]
		\centering
		\includegraphics[width=0.446\linewidth]{Figures/Dev-SOFT/app_state_logging}
		\caption{Fowchart state logging}
		\label{fig:appstatelogging}
	\end{figure}

	\clearpage
	

}


% ---------- MESURE PREUVE DE CONCEPT ----------------------------- 
\section{Validation du design}
Lors de cette section, sera décrite la procédure de vérification des caractéristiques du projet ainsi la validation de celui-ci.
\subsection{Liste de matériel} \label{ssec:ListeMateriel}
{
	\begin{enumerate}
		\item Oscilloscope Tektronix RTB2004 ES.SLO2.05.01.16 \label{enum:oscillo}
		\item USB Logic Analyzer, 8-Canaux, 24MHz \label{enum:logicAnalyzer}
		\item USB TO TLL HW-597 \label{enum:USB-TTL}
		\item Carte Localisation-Sous-Marine V0.0 \label{enum:PCBL}
	\end{enumerate}
}

\subsection{Contrôle des alimentations}
{
	En premier lieu, une vérification des tensions d'alimentation permet de valider un aspect critique et fondamentale de la carte.
	\subsubsection{Méthodologie}
	\paragraph{Mesure du 3.3V :} Alimentation de la carte par une connections brève entre les pins du connecteur du bouton \textbf{P15}. Mesure sur le testpoint \textit{V\_regOUT5} voir figure \ref{fig:sch3}.
	\begin{figure}[h]
		\centering
		\includegraphics[width=0.3\linewidth]{Figures/DEV_MEAS/Sch3.3V}
		\caption{Testpoint mesure 3.3V}
		\label{fig:sch3}
	\end{figure}
	
	\paragraph{Mesure du 5V :} Pour mesurer le 5V, il faut ponter par une résistance $0\Omega$ la résistance R50, ainsi que activer la Pin RC5 / EN\_5V. Ensuite la mesure a été prise sur le connecteur P16, pin-3 :
	\begin{figure}[h]
		\centering
		\includegraphics[width=0.2\linewidth]{Figures/DEV_MEAS/Sch5V}
		\caption{Schéma de mesure 5V}
		\label{fig:sch5v}
	\end{figure}

	\clearpage
	
	\subsubsection{Mesures}
	
	\begin{figure}
	\begin{subfigure}{.5\textwidth}
		\centering
		\includegraphics[width=\textwidth]{Mesures/Tension3.3V}
		\caption{Mesure 3.3V}
		\label{fig:Mes3.3V}
	\end{subfigure}
	\begin{subfigure}{.5\textwidth}
		\centering
		\includegraphics[width=\textwidth]{Mesures/Tension5V}
		\caption{Mesure 5V}
		\label{fig:Mes5V}
	\end{subfigure}
	\caption{Mesures des tensions d'alimentations}
	\label{fig:Mesure3.3et5V}
	\end{figure}
	
	\paragraph{Analyse :} Nous pouvons observer les valeurs respective des tensions sur les figures \ref{fig:Mes3.3V} et \ref{fig:Mes5V}. La tension d'alimentation du microcontrôleur est mesurée à \textit{3.125V} ce qui peut être expliqué par une chute de tension aux bornes de la batterie LI-ION qui par conséquent à besoin d'être chargée. La tension 3.3V oscille légèrement, ceci peut s'expliquer par le fait qu'il s'agit de l'alimentation principale de la carte avec le plus de consommateurs. Mais également par le fait que des signaux rapides sont générés par le microcontrôleur et ses différents périphériques.
	
	La tension d'alimentation du capteur de pression dont la tension dimensionnée est de 5V est quant-à-elle mesurée à \textit{5.0295V} ce qui signifie une bonne précision de la part du circuit de boost. On peut également voir que la tension n'oscille pas et ne vas donc pas perturber le capteur de pression.
	
	Nous pouvons donc confirmer le fonctionnement des blocs d'alimentation du projet.
	
}

\subsection{Communication UART}
{
	Comme décrit lors de la sous-section \ref{ssec:Usart} une communication série est implémenté dans projet. Cette communication n'est pas critique pour le projet mais il est important de vérifier son fonctionnement pour d'éventuelles versions ultérieures.
	\subsubsection{Méthodologie}
	Pour la mesure, l'analyseur logique (Numéro \ref{enum:logicAnalyzer} de la liste de matériel \ref{ssec:ListeMateriel}) a été utilisé. Les trames U1TX et U1RX ont été mesurée sur le connecteur P4 :
	
	\begin{figure}[h]
		\centering
		\includegraphics[width=0.2\linewidth]{Figures/DEV_MEAS/SchUart1}
		\caption{Schéma de mesure UART1}
		\label{fig:schuart1}
	\end{figure}
	\clearpage	
	\subsubsection{Mesures}
	
	\begin{figure}[h]
		\centering
		\includegraphics[width=1.1\linewidth]{Mesures/MesureTrameUart_Upscaled}
		\caption{Mesures trames UART}
		\label{fig:mesuretrameuart}
	\end{figure}

	\paragraph{Analyse :} On peut lire en caractères ascii sur la trame de la figure \ref{fig:mesuretrameuart} : 
	\begin{lstlisting}[frame=single, caption={Trame UART en ASCII}, captionpos=b, breaklines=true]
TIME: 7 ms 
Gyro    : X = 0.000	Y 
	\end{lstlisting} 

	En connectant le module USB-TO-TTL (Numéro \ref{enum:USB-TTL} de la liste de matériel \ref{ssec:ListeMateriel}) au broches Tx et Rx de la figure \ref{fig:mesuretrameuart}. Puis en ouvrant une communication série via PuTTY à un bauderate 115200. On obtient la communication de la figure \ref{} dans la console :
	
	\begin{figure}[h]
		\centering
		\fbox{\includegraphics[width=0.7\linewidth]{Mesures/PuttYSerComm}}
		\caption{Récéption UART PuTTY}
		\label{fig:puttysercomm}
	\end{figure}
	
	

}

\subsection{Communication SPI, carte SD}
{
	\subsubsection{Méthodologie}
	
	\subsubsection{Mesures}

}

\section{Caractéristiques du produit finis}

% J'ai pus sans difficulté faire un logging sur 5h et ai récolté 30MB de données dans le fichier CSV, ce qui correspond aux $\sim5$ heures de logging


% ---- Bibliographie ----
%\input{bibliography}

\clearpage

\section{Conclusion}
{
	\paragraph{Synthèse} Le présent rapport a décrit le développement et la validation d'un projet de localisation sous-marine. L'objectif principal du projet était de concevoir et de mettre en oeuvre un système capable de collecter et de stocker des données de déplacement, de temps et de pression lors de plongées. Après avoir réalisé les différentes étapes de développement, nous pouvons conclure que le projet a été mené à une version finie.
	
	\paragraph{Développement} Au cours de la phase de développement, plusieurs étapes clés ont été franchies. Tout d'abord, une analyse approfondie des besoins et des contraintes a été réalisée, ce qui a permis de définir les spécifications du système. Ensuite, un processus itératif de conception a été suivi, comprenant la sélection des composants appropriés, la création des schémas électroniques, et la fabrication du prototype.
	
	\paragraph{Design} L'évaluation du design du projet a été effectuée en suivant une méthodologie rigoureuse de vérification et de validation. Les principales caractéristiques du projet, telles que les tensions d'alimentation, la communication UART et la communication SPI avec la carte SD, ont été vérifiées avec succès. Les mesures effectuées ont montré que le système fonctionnait correctement et répondait aux spécifications requises.
	
	\paragraph{Test} Le projet a été testé avec succès lors d'un enregistrement de données de déplacement pendant une durée de 5 heures, ce qui a permis de collecter 30 MB de données. Ces résultats confirment que le système est capable de fonctionner de manière fiable et de fournir les fonctionnalités attendues.
	
	\paragraph{Apports} Ce projet a permis d'acquérir une expérience précieuse dans le domaine de la conception et du développement de systèmes électroniques. Il a également mis en évidence l'importance de l'organisation, de la structure et de la vérification étape par étape des éléments du design.
	
	\paragraph{Correctifs} Afin de simplifier la mise en place du système, des correctifs mentionnés dans le fichier MODIF permettrait de palier a certaines erreurs non-critiques de développements.
	
	\paragraph{Améliorations} Des améliorations et des développements futurs peuvent être envisagés, tels que l'ajout de fonctionnalités supplémentaires, l'optimisation de la communication, l'extension des capacités de stockage ainsi que la mise en place d'une communication USB directement par le FTDI en corrigeant le pinning de SCK. Ces évolutions permettraient d'explorer de nouvelles possibilités d'application de ce système de localisation sous-marine.
}


\newpage
\nocite{*}
\section{Bibliographie}
\bibliography{Biblio-Proj} 
\bibliographystyle{ieeetr}

% ANNEXES
\clearpage
\section{Annexes}
\includepdf[scale=0.8,pages=1,pagecommand=\subsection{Bill of materials}]{../../bom/Bill of Materials.pdf}
\includepdf[scale=0.8,pages=2]{../../bom/Bill of Materials.pdf}

\end{document}
